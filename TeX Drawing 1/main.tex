\documentclass[12pt,english]{article}
\usepackage[T1]{fontenc}
\usepackage[latin9]{inputenc}
\usepackage{amsmath}
\usepackage{amssymb}
\usepackage{enumerate, mathrsfs}
\usepackage{pstricks-add, pst-plot}
\usepackage{auto-pst-pdf}

\makeatletter

\usepackage[letterpaper,body={6.5in,9in}, head=34pt, foot=70pt]{geometry}
\usepackage{fancyhdr}

\pdfpagewidth 8.5in
\pdfpageheight 11in

\pagestyle{fancy}
\headheight 35pt

\rhead{}
\chead{}
\lhead{}
\renewcommand{\footrulewidth}{0.4pt}% default is 0pt
\rfoot{\thepage}
\cfoot{}
\lfoot{}

\usepackage{multicol}

\makeatother

\usepackage{babel}
\begin{document}

\begin{enumerate}
\item Consider the following linear programming problem
\begin{align*}
\text{minimize}\ \ 4y_{1}+7y_{2}\\
\text{subject to }\ \ 2y_{1}+y_{2} & \geq5\\
\ 3y_{1}+2y_{2} & \geq2\\
\ y_{1}+3y_{2} & \geq5\\
y_{1},y_{2} & \geq0
\end{align*}
Solve this problem using the graphical method. (Hint: determine the feasible region, and evaluate the corner points using algebra. Notice this is a minimization problem)

\vskip 2ex
We can find our feasible region by graphing the equalities portion of the inequalities.
\psset{unit=1.4cm}
%(y1,y2)
\begin{center}
    \begin{pspicture}[showgrid](0,0)(8,8)
        %Axis
        \psline[linewidth=1.5pt]{<->}(0,-0.3)(0,8.2)
        \psline[linewidth=1.5pt]{<->}(-0.3,0)(8.2,0)
        %Magenta line & fill, 2y1 + y2 >= 5
        \psline[linewidth=1.5pt,linecolor=magenta!75]{<->}(0,5)(2.5,0)
        %Black line & fill, 2y1 + y2 >= 5
        \psline[linewidth=1.5pt,linecolor=black!75]{<->}(0,1)(0.6666666667,0)
        %Orange line & fill, 2y1 + y2 >= 5
        \psline[linewidth=1.5pt,linecolor=orange!75]{<->}(0,1.6666666667)(5,0)
        %Blue line & fill, y2 >= 0
        \psline[linewidth=1.5pt,linecolor=blue!75]{<->}(0,-0.2)(0,8.1)
        %Red line, y1 >= 0
        \psline[linewidth=1.5pt,linecolor=red!75]{<->}(-0.2,0)(8.1,0)
    \end{pspicture}
\vskip 2ex
Since everything is greater than or equal to, we know the feasablie region is above all the lines, so our solution is inside the following region:

    \begin{pspicture}[showgrid](0,0)(8,8)
        %Axis
        \psline[linewidth=1.5pt]{<->}(0,-0.3)(0,8.2)
        \psline[linewidth=1.5pt]{<->}(-0.3,0)(8.2,0)
        %Magenta line & fill, 2y1 + y2 >= 5
        \psline[linewidth=1.5pt,linecolor=magenta!75](0,5)(2,1)
        %Orange line & fill, 2y1 + y2 >= 5
        \psline[linewidth=1.5pt,linecolor=orange!75](2,1)(5,0)
        %Blue line & fill, y2 >= 0
        \psline[linewidth=1.5pt,linecolor=blue!75]{->}(0,5)(0,8.1)
        %Red line, y1 >= 0
        \psline[linewidth=1.5pt,linecolor=red!75]{->}(5,0)(8.1,0)
        \rput(0,5.3){Corner 1}
        \pscircle[fillstyle=solid, fillcolor=black](0,5){.075}
        \rput(2,1.3){Corner 2}
        \pscircle[fillstyle=solid, fillcolor=black](2,1){.075}
        \rput(5,0.3){Corner 3}
        \pscircle[fillstyle=solid, fillcolor=black](5,0){.075}
    \end{pspicture}
\vskip 2ex


\vskip 2ex
\end{center}
From here we can test the corner points of our feasible region, which occur at: $(0,5)$, $(2,1)$, and $(5,0)$. Substituting: $4(5)+7(0)=20$, $4(2)+7(1)=15$, and $4(0)+7(5)=35$, $min(20,15,35)=15$ thus the minimized solution is $\boxed{15}$
\pagebreak
\item Consider the following vectors and matrix:
\begin{align*}
\mathbf{x}=\left[\begin{array}{c}
x_{1}\\
x_{2}\\
x_{3}\\
x_{4}\\
x_{5}
\end{array}\right] & \quad\mathbf{c}=\left[\begin{array}{c}
1\\
5\\
3\\
7\\
5
\end{array}\right]\quad A=\left[\begin{array}{ccccc}
1 & 0 & -2 & 4 & 5\\
3 & 1 & 1 & 2 & 4\\
0 & 1 & -1 & 2 & 2
\end{array}\right]
\end{align*}
Perform the following operations:

\begin{enumerate}
\item $\mathbf{c}^{T}\mathbf{x}$\textbf{\vspace{1cm}}
$$
\mathbf{c}^{T} =
\begin{bmatrix}
1 & 5 & 3 & 7 & 5
\end{bmatrix}
$$
$$
\mathbf{c}^{T}\mathbf{x} = 
\begin{bmatrix}
1 & 5 & 3 & 7 & 5
\end{bmatrix}
\times
\begin{bmatrix}
x_{1}\\
x_{2}\\
x_{3}\\
x_{4}\\
x_{5}
\end{bmatrix}
=\boxed{
\begin{bmatrix}
x_1 & 5x_2 & 3x_3 & 7x_4 & 5x_5
\end{bmatrix}}
$$
\item $A\mathbf{c}$\textbf{\vspace{1cm}}

$$
\begin{bmatrix}
1 & 0 & -2 & 4 & 5\\
3 & 1 & 1 & 2 & 4\\
0 & 1 & -1 & 2 & 2
\end{bmatrix}
\times
\begin{bmatrix}
1\\
5\\
3\\
7\\
5
\end{bmatrix}
=
\begin{bmatrix}
1+0-6+28+25\\
3+5+3+14+20\\
0+5-3+14+10
\end{bmatrix}
=\boxed{
\begin{bmatrix}
48\\
45\\
26
\end{bmatrix}}
$$

\end{enumerate}
\pagebreak
\item Consider the following optimization problem:
\begin{align*}
\text{maximize}\ \ 5x_{1}+2x_{2}+5x_{3}\\
\text{subject to }\ \ 2x_{1}+3x_{2}+x_{3} & \leq4\\
\ x_{1}+2x_{2}+3x_{3} & \leq7\\
x_{1},x_{2},x_{3} & \geq0
\end{align*}
Write the problem above in matrix-vector notation.
\vskip 2ex
First we must rewrite the conditions of the problem, for all feasible solutions: $x_1, x_2, x_3$, the value on the LHS is at most the value on the RHS, so we have that:

$$ 2x_1 + 3x_2 + x_3 + s_1 = 4 $$
$$ x_1 + 2x_2 + 3x_3 + s_2 = 7 $$

Since we are maximizing, the solution function becomes:

$$5x_1-2x_2-5x_3+z=0$$

Thus, the final matrix is:
$$\boxed{
\begin{bmatrix}
x_1 & x_2 & x_3 & s_1 & s_2 & z &\vline& C\\
\hline
2 & 3 & 1 & 1 & 0 & 0 & \vline & 4\\
1 & 2 & 3 & 0 & 1 & 0 & \vline & 7\\
-5 & -2 & -5 & 0 & 0 & 1 & \vline & 0
\end{bmatrix}}
$$
\end{enumerate}

\end{document}
