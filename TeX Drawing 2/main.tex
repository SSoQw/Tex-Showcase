\documentclass[11 pt, letterpaper]{exam}
%\printanswers
\usepackage{amsfonts}
\usepackage{graphicx}
\usepackage{amsthm}
\usepackage{amssymb}
\usepackage{amsmath}
\usepackage{csvsimple}
\usepackage{enumerate, mathrsfs}
\usepackage{pstricks-add, pst-plot}
\usepackage{auto-pst-pdf}

\newcommand{\suchthat}{\,\Big{|}\,}
\newcommand{\ZZ}{\mathbb{Z}}
\newcommand{\QQ}{\mathbb{Q}}
\newcommand{\NN}{\mathbb{N}}
\newcommand{\RR}{\mathbb{R}}
\newcommand{\CC}{\mathbb{C}}



\firstpageheader{}{}{}
\begin{document}

\begin{questions}
\question Consider the function $f(x) = \sqrt{5x}$ on $[4,6]$.  It is clear that $p=5$ is a fixed point of $f$.  
\begin{parts}
\part[10] Check all of the conditions of Theorem 2.8 and justify a conclusion that the iterative sequence $\{p_n\}$ (for $p_0\in [4,6]$ and for $p_0\not=5$) converges linearly to $p=5$.
\vskip 2ex

First we must check that $f(x)$ is continuous on $[4,6]$, such that $\forall x \in[4,6], f(x)\in[4,6]$. 
\vskip 1ex
\begin{center}
$f(x) = \sqrt{5x}$
\vskip 2ex
    \begin{tabular}{c c c}
        $f(4) = \sqrt{5\times4}$ & \text{    }& $f(6) = \sqrt{5\times6}$ \\
          $ = 2\sqrt{5}$ & & $ = \sqrt{30}$ \\
           & $f(x) \in [2\sqrt{5},\sqrt{30}]$ & \\
    \end{tabular}\\
\end{center}
\vskip 1ex
$$\boxed{\therefore \forall x \in[4,6],~f(x) \in [4,6]}$$

\vskip 2ex
\psset{unit=3.2cm}
\begin{center}
    \begin{pspicture}[showgrid](3,3)(7,7)
        %Blue line, f(x) = sqrt(5x)
        \psplot[linewidth=1.5pt,linecolor=blue,algebraic]{+4}{+6}{sqrt(5*x)}
        \psplot[linestyle=dashed, linewidth=1.5pt,linecolor=blue,algebraic]{+3}{+4}{sqrt(5*x)}
        \psplot[linestyle=dashed, linewidth=1.5pt,linecolor=blue,algebraic]{+6}{+7}{sqrt(5*x)}
        %Red line, f(x) = x
        \psline[linestyle=dashed, linewidth=1.5pt, linecolor=red](3,3)(4,4)
        \psline[linestyle=dashed, linewidth=1.5pt, linecolor=red](6,6)(7,7)
        \psline[linewidth=1.5pt, linecolor=red](4,4)(6,6)
        %Black Box
        \psline[linewidth=1.5pt](4,4)(4,6)
        \psline[linewidth=1.5pt](4,4)(6,4)
        \psline[linewidth=1.5pt](4,6)(6,6)
        \psline[linewidth=1.5pt](6,4)(6,6) 
    \end{pspicture}
\vskip 2ex
The blue line is the function $f(x) = \sqrt{5x}$ and the red line is the function $f(x)=x$
\vskip 2ex
\end{center}
A we will see later, the derivative of thus function is never less than zero on [4,6], the function is continous and increasing. Thus the first condition of Theorem 2.8 is met.
\pagebreak

Now we must find a positive constant $k \suchthat k<1 \wedge \forall x\in(4,6),~\mid f'(x)\mid \leq k$. 

\begin{center}
$f'(x) = \frac{\sqrt{5}}{2\sqrt{x}}$
\vskip 2ex
    \begin{tabular}{c c c}
        $f'(4) = \frac{\sqrt{5}}{2\sqrt{4}}$ & \text{    }& $f'(6) = \frac{\sqrt{5}}{2\sqrt{6}}$ \\
          $ = \frac{\sqrt{5}}{4}$ & & $ = \frac{\sqrt{30}}{12}$ \\
           & $f'(x) \in (\frac{\sqrt{30}}{12},\frac{\sqrt{5}}{4})$ &
    \end{tabular}\\
\end{center}

Thus, 
$$\frac{\sqrt{30}}{12}<f'(x)<\frac{\sqrt{5}}{4}\leq k<1$$

$$\boxed{\therefore  k \in [\frac{\sqrt{5}}{4},1)}$$
\vskip 2ex
\psset{unit=3.75cm}
\begin{center}
    \begin{pspicture}[showgrid](3,-1)(7,2)
        %Blue line, f'(x) = sqrt(5)/(2*sqrt(x))
        \psplot[linewidth=1.5pt,linecolor=blue,algebraic]{+4}{+6}{sqrt(5)/(2*sqrt(x))}
        \psplot[linestyle=dashed, linewidth=1.5pt,linecolor=blue,algebraic]{+3}{+4}{sqrt(5)/(2*sqrt(x))}
        \psplot[linestyle=dashed, linewidth=1.5pt,linecolor=blue,algebraic]{+6}{+7}{sqrt(5)/(2*sqrt(x))}
        %Fill for inequality
        \pscustom[fillstyle=solid,fillcolor=red!75,linestyle=none,algebraic]{
        \psline(4,0.559016994)(4,1)
        \psline(6,1)(6,0.559016994)
        }
        \psline[linewidth=1.5pt, linecolor=red](4,0.559016994)(6,0.559016994)
        \psline[linestyle=dashed,
        linewidth=1.5pt,linecolor=red](6,0.559016994)(7,0.559016994)
        \psline[linestyle=dashed, linewidth=1.5pt,linecolor=red](3,0.559016994)(4,0.559016994)
        %Black Box
        \psline[linewidth=1.5pt](4,0)(4,1)
        \psline[linewidth=1.5pt](4,0)(6,0)
        \psline[linewidth=1.5pt](4,1)(6,1)
        \psline[linewidth=1.5pt](6,0)(6,1)
    \end{pspicture}
\vskip 2ex
The blue line is the function $f'(x)$ and the red shaded area is $\mid f'(x) \mid \leq k < 1$.
\end{center}
\vskip 1ex
Thus the second condition of Theorem 2.8 is met. Since $\forall x \in [4,6],~f'(x) \neq 0$ any $x\in [4,6],~x\neq5$ will converge linearly to unique fixed point $x=5$.  
\pagebreak

\part[10] Generate a table including values of $p_i$ and $\frac{|p_i-5|}{|p_{i-1}-5|^1}$ for $0\leq i\leq 20$.
\begin{center}
    \csvautotabular{T3.txt}
\end{center}
\part[5] Estimate the asymptotic error constant $\lambda$.
$$\boxed{\lambda = \frac{1}{2}}$$
\end{parts}
\question Show that the sequences below converge linearly to $p=0$.  How many terms are required before $|p_n-p|<5\times 10^{-2}$?
\begin{parts}
\part[10] $\displaystyle{p_n=\frac{1}{n}}$
$$\lim_{n\to\infty} \frac{\mid p_{n+1}\mid}{\mid p_n \mid^{\alpha}} = \lim_{n\to\infty}\frac{\mid \frac{1}{n+1}\mid}{\mid \frac{1}{n} \mid^{\alpha}} = \lim_{n\to\infty} \frac{n^\alpha}{n+1}$$
We will have convergence for $\alpha = 1$, but not $\alpha = 2$ so this sequence will converge linearly to 0.
\vskip 1ex
Terms to be within $5\times 10^{-2}$
$$\mid p_n-p \mid < 5\times10^{-2} \rightarrow \frac{1}{n}<5\times10^{-2} \rightarrow \boxed{n > 20}$$
\pagebreak
\part[10] $\displaystyle{p_n=\frac{1}{n^2}}$
$$\lim_{n\to\infty} \frac{\mid p_{n+1}\mid}{\mid p_n \mid^{\alpha}} = \lim_{n\to\infty}\frac{\mid \frac{1}{(n+1)^2}\mid}{\mid \frac{1}{n^2} \mid^{\alpha}} = \lim_{n\to\infty} (\frac{n^\alpha}{n+1})^2$$
Similarly to part a, will have convergence for $\alpha = 1$, but not $\alpha = 2$ so this sequence will also converge linearly to 0.
\vskip 1ex
Terms to be within $5\times 10^{-2}$
$$\mid p_n-p \mid < 5\times10^{-2} \rightarrow \frac{1}{n^2}<5\times10^{-2} \rightarrow 20 < n^2 \rightarrow \boxed{n > \sqrt{20} \approx 4.47214}$$
\end{parts}
\question[10] Show that $\displaystyle{p_n=\left(\frac{1}{10}\right)^{2^{^n}}}$ converges to 0 quadratically.
$$\lim_{n\to\infty} \frac{\mid p_{n+1}\mid}{\mid p_n \mid^{\alpha}} = \lim_{n\to\infty} \frac{\mid 10^{-2^{n+1}}\mid}{\mid 10^{-2^{n}} \mid^{\alpha}} = \lim_{n\to\infty} \frac{10^{-2^{n}2}}{10^{-2^{n}\alpha}}$$
We will have convergence for $\alpha = 1$ as well as $\alpha = 2$, but not $\alpha = 3$, so this this sequence will quadratically to 0.
\question 
\begin{parts}
\part[10] Write down the formula for a sequence $p_n$ that converges to $p=0$ with order $\alpha=3$.
$$p_n = 100^{-3^{n}}$$

\part[5] Generate the first 5 terms of this sequence.
\begin{center}
    \begin{tabular}{|c | c|}
        \hline
        $n$ & $p_n$ \\
        \hline
        $0$ & $1\times 10^{-2}$ \\
        $1$ & $1\times 10^{-6}$ \\
        $2$ & $1\times 10^{-18}$ \\
        $3$ & $1\times 10^{-54}$ \\
        $4$ & $1\times 10^{-162}$ \\
        \hline
    \end{tabular}
\end{center}
\end{parts}
\end{questions}
\end{document}